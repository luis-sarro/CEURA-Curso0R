\section{Datos}

\subsection{Lectura de datos}

\begin{itemize}
    \item Para leer un fichero simple, con los datos separados
    por espacios en blanco, tabuladores o saltos de línea, se
    utiliza la instrucción \verb"read.table" en la forma:
    \begin{verbatim}
    > fichero.df <- read.table("c:/dir/mi.fichero",
    + header = TRUE, sep = "",
    + comment.char = "")
    \end{verbatim}

    \item Si el carácter decimal no es un punto sino, por ej.,
    una coma, usar: \verb[dec = ","[.

    \item Se pueden saltar líneas (\verb"skip") o leer un número
    fijo de líneas (\verb"nrows").

    \item Hay funciones especializadas para otros archivos
    (ej., \verb"read.csv") pero son casos específicos de \verb"read.table".

\end{itemize}

\newslide

\subsection{Guardar y leer datos}

\begin{itemize}
    \item Resulta muy importante poder guardar datos, funciones, etc., para
    ser usados en otras sesiones de R. Esos datos así guardados pueden
    compartirse con otros usuarios e incluso utilizarse en distintos
    sistemas operativos.
    \begin{verbatim}
    > x <- runif(20)
    > y <- list(a = 1, b = TRUE, c = "patata")
    > save(x, y, file = "xy.RData")
    \end{verbatim}
    \item Los leeremos con
    \begin{verbatim}
    > load("xy.RData")
    \end{verbatim}

\newslide

    \item Podemos guardar todos los objetos con
    \begin{verbatim}
    > save.image() # guardado como ".RData"
    > save.image(file = "nombre.RData")
    \end{verbatim}
    \item  El fichero \verb".RData" se carga al iniciarse R.
    
    \item R y muchos otros paquetes incorporan archivos con datos:

    Se cargan con \verb[load("nombre.RData")[.

    \item La instrucción \verb"data" permite cargar archivos
    de las librerías disponibles.
    \begin{verbatim}
    > data() # muestra todos los archivos
    > data(iris)
    > data(iris, package = "base") # equivalente
    > ?iris
    \end{verbatim}
\end{itemize}

\subsection{Importar datos de Excel}

\begin{itemize}
    \item Lo mejor es exportar los datos desde Excel a un archivo
    de texto separado por tabuladores.
    \item Cuidado con las últimas columnas y \emph{missing data}
    (Excel elimina los ``trailing tabs''). Dos formas de minimizar
    problemas: \begin{itemize}
        \item Usar \verb"NA" para missing.
        \item Poner una última columna con datos arbitrarios (ej.,
        una columna llena de 2s).
    \end{itemize}
    \item Cuidado también con líneas extra al final del fichero.
    \item Salvamos como texto (sólo salvamos una de las hojas).
    \item Importamos en R con \verb"read.table".
\end{itemize}

\newslide

\subsection{Exportar datos}

\begin{itemize}
    \item Lo más sencillo es exportar una matriz

    (es necesario transponer la matriz).
    \begin{verbatim}
    > write(t(x), file = "c:/dir/data.txt",
    +       ncolumns = n,
    +       append = FALSE)
    \end{verbatim}

    \item Pero para exportar un \verb"data.frame" es mejor
    \begin{verbatim}
    > write.table(my.data.frame,
    +       file = "mi.output.txt",
    +       sep = "",row.names = FALSE,
    +       col.names = TRUE)
    \end{verbatim}
    \item  Para escribir un fichero CSV importable desde Excel
    \begin{verbatim}
    >  write.table(x, file = "foo.csv", sep = ",",
    +              col.names = NA)
    \end{verbatim}
\end{itemize}

%\newslide



\section{Gráficos}

\subsection{Introducción}

\begin{itemize}
\item R incluye muchas y variadas funciones para hacer gráficos.
\item El sistema permite desde gráficos muy simples a figuras de
calidad para incluir en artículos y libros.
\item Sólo examinaremos la superficie. Más detalles en el libro \emph{R Graphics} de Paul Murrell.
\item También podemos ver un buen conjunto de ejemplos con \verb"demo(graphics)".
\item El comando \verb"plot" es uno de los más utilizados para realizar gráficos.
\end{itemize}

\newslide

\subsection{El comando \texttt{plot}}

\begin{itemize}
\item Si escribimos \verb"plot"$(\mathbf{x},\mathbf{y})$ donde $\mathbf{x}$ e $\mathbf{y}$ son vectores con $n$ coordenadas, entonces R representa el gráfico de dispersión
con los puntos de coordenadas $(x_i,y_i)$.
\item Este comando incluye por defecto una elección automática de ejes,
escalas, etiquetas de los ejes, densidad de las líneas, etc., que pueden
ser modificados añadiendo parámetros gráficos al comando y que pueden
visualizarse con \verb"help(par)".
\begin{verbatim}
> x <- runif(50, 0, 4); y <- runif(50, 0, 4)
> plot(x, y, main = "Título principal",
+ sub = "subtítulo", xlab = "eje x", ylab = "eje y",
+ xlim = c(-5,5),ylim = c (-5,5))
\end{verbatim}

\newslide

\item Variaciones de \verb"plot":
\begin{verbatim}
> z <- cbind(x,y)
> plot(z)
> plot(y ~ x)
> plot(log(y + 1) ~ x) # transformación de y
> plot(x, y, type = "p")
> plot(x, y, type = "l")
> plot(x, y, type = "b")
> plot(c(1,5), c(1,5))
> legend(1, 4, c("uno", "dos", "tres"), lty = 1:3,
+ col = c("red", "blue", "green"),
+ pch = 15:17, cex = 2)
\end{verbatim}
\item Con \verb"text" podemos representar caracteres de texto directamente:
\begin{verbatim}
> sexo <- c(rep("v", 20), rep("m", 30))
> plot(x, y, type = "n")
> text(x, y, labels = sexo)
\end{verbatim}

\newslide

\item Puntos.
\begin{verbatim}
> points(x, y, pch = 3, col = "red")
\end{verbatim}
\item Tipos de puntos.
\begin{verbatim}
> plot(c(1, 10), c(1, 3), type = "n", axes = FALSE,
+      xlab = "", ylab="")
> points(1:10, rep(1, 10), pch = 1:10, cex = 2, col = "blue")
> points(1:10, rep(2, 10), pch = 11:20, cex = 2, col = "red")
> points(1:10, rep(3, 10), pch = 21:30, cex = 2,
+        col = "blue", bg = "yellow")
\end{verbatim}
\item Tipos de líneas.
\begin{verbatim}
> plot(c(0, 10), c(0, 10), type = "n", xlab ="",
+ ylab ="")
> for(i in 1:10)
+ abline(0, i/5, lty = i, lwd = 2)
> for(i in 1:10)
+ abline(0, i/5, lty = i, lwd = 2)
\end{verbatim}

\newslide

\item \verb"lty" permite especificaciones más complejas
(longitud de los segmentos que son alternativamente dibujados y no dibujados).
\item \verb"par" controla muchos parámetros gráficos.
Por ejemplo, \verb"cex" puede referirse a los ``labels'' (\verb"cex.lab"),
otro, \verb"cex.axis", a la anotación de los ejes, etc.
\item Hay muchos más colores. Ver \verb"palette", \verb"colors".

\end{itemize}

\newslide

\subsection{Identificación interactiva de datos}

\begin{itemize}
\item \verb"identify"$(x,y,\textit{etiquetas})$ identifica los puntos con el ratón
y escribe la correspondiente etiqueta.
\begin{verbatim}
> x <- 1:10
> y <- sample(1:10)
> nombres <- paste("punto", x, ".", y, sep ="")
> plot(x, y)
> identify(x, y, labels = nombres)
\end{verbatim}
\item \verb"locator()" devuelve las coordenadas de los puntos.
\begin{verbatim}
> plot(x, y)
> locator()
> text(locator(1), "el marcado", adj = 0)
\end{verbatim}
\end{itemize}

\newslide

\subsection{Múltiples gráficos por ventana}

\begin{itemize}
\item Empezamos con \verb"par(mfrow=c("\textit{filas},\textit{columnas}\verb"))"
antes del comando \verb"plot".
\begin{verbatim}
> par(mfrow = c(2, 2))
> plot(rnorm(10))
> plot(runif(5), rnorm(5))
> plot(runif(10))
> plot(rnorm(10), rnorm(10))
\end{verbatim}
\item Podemos mostrar muchos gráficos en el mismo dispositivo gráfico.
La función más flexible y sofisticada es \verb"split.screen",
bien explicada en \emph{R para principiantes}, secc. 4.1.2 (p. 30).
\end{itemize}

\newslide

\subsection{Datos multivariantes}

\begin{itemize}
\item Diagrama de dispersión múltiple.
\begin{verbatim}
> X <- matrix(rnorm(1000), ncol = 5)
> colnames(X) <- c("a", "id", "edad", "loc",
+ "weight")
> pairs(X)
\end{verbatim}
\item Gráficos condicionados (revelan interacciones).
\begin{verbatim}
> Y <- as.data.frame(X)
> Y$sexo <- as.factor(c(rep("Macho", 80),
+ rep("Hembra", 120)))
> coplot(weight ~ edad | sexo, data = Y)
> coplot(weight ~ edad | loc, data = Y)
> coplot(weight ~ edad | loc * sexo, data = Y)
\end{verbatim}
\item La librería \verb"lattice" permite lo mismo, y mucho más, que
\verb"coplot". Ver secc. 4.6 de \emph{R para principiantes}.
\end{itemize}

\newslide

\subsection{Boxplots}

\begin{itemize}
\item Los diagramas de caja son muy útiles para ver rápidamente
las principales características de una variable cuantitativa,
o comparar entre variables.
\begin{verbatim}
> attach(Y)
> boxplot(weight)
> plot(sexo, weight)
> detach()
> boxplot(weight ~ sexo, data = Y,
+ col = c("red", "blue"))
\end{verbatim}
\item La función \verb"boxplot" tiene muchas opciones;
se puede modificar el aspecto, mostrarlos horizontalmente,
en una matriz de boxplots, etc. Véase la ayuda \verb"?boxplot".
\end{itemize}

\newslide

\subsection{Un poco de ruido}

Los datos cuantitativos discretos pueden ser difíciles de ver bien.
Podemos añadir un poco de ruido con el comando \verb"jitter".
\begin{verbatim}
> dc1 <- sample(1:5, 500, replace = TRUE)
> dc2 <- dc1 + sample(-2:2, 500, replace = TRUE,
+ prob = c(1, 2, 3, 2, 1)/9)
> plot(dc1, dc2)
> plot(jitter(dc1), jitter(dc2))
\end{verbatim}

\newslide

\subsection{Dibujar rectas}

\begin{itemize}
\item Podemos añadir muchos elementos a un gráfico, además
de leyendas y líneas rectas.
\begin{verbatim}
> x <- rnorm(50)
> y <- rnorm(50)
> plot(x, y)
> lines(lowess(x, y), lty = 2)
> plot(x, y)
> abline(lm(y ~ x), lty = 3)
\end{verbatim}
\item Podemos añadir otros elementos con ``panel functions''
en otras funciones (como \verb"pairs", \verb"lattice", etc).
\end{itemize}

\newslide

\subsection{Más gráficos}

\begin{itemize}
\item Podemos modificar márgenes exteriores de figuras y entre
figuras (véase \verb"?par" y búsquense \verb"oma", \verb"omi",
\verb"mar", \verb"mai"; ejemplos en \emph{An introduction to R},
secc. 12.5.3 y 12.5.4.

\item También gráficos 3D: \verb"persp", \verb"image",
\verb"contour"; histogramas: \verb"hist"; gráficos de barras:
\verb"barplot"; gráficos de comparación de cuantiles, usados para
comparar la distribución de dos variables, o la disribución de
unos datos frente a un estándar (ej., distribución normal):
\verb"qqplot", \verb"qqnorm" y, en paquete \verb"car",
\verb"qq.plot".

\item Notación matemática (\verb"plotmath") y expresiones de texto
arbitrariamente complejas.

\item Gráficos tridimensionales dinámicos con XGobi y GGobi. Ver:
\url{http://cran.r-project.org/src/contrib/Descriptions/xgobi.html},
\url{http://www.ggobi.org},
\url{http://www.mcs.vuw.ac.nz/~ray/R-stuff/windows/gguide.pdf}.
\end{itemize}

\newslide

\subsection{Guardar los gráficos}

\begin{itemize}
\item En Windows, podemos usar los menús y guardar con distintos
formatos.

\item También podemos especificar donde queremos guardar el
gráfico.
\begin{verbatim}
> pdf(file = "f1.pdf", width = 8, height = 10)
> plot(rnorm(10))
> dev.off()
\end{verbatim}

\item O bien, podemos copiar una figura a un fichero.
\begin{verbatim}
> plot(runif(50))
> dev.copy2eps()
\end{verbatim}

\end{itemize}

\newslide



\endinput
