\section{Funciones}

\subsection{Definición de funciones}

\begin{itemize}
    \item R es un lenguaje que permite crear nuevas funciones.
    Una función se define con una asignación de la forma
    \begin{verbatim}
    > nombre <- function(arg1,arg2,...){expresión}
    \end{verbatim}

    \item La expresión es una fórmula o grupo de fórmulas
    que utilizan los argumentos para calcular su valor.
    El valor de dicha expresión es el valor que proporciona
    R en su salida y éste puede ser un simple número, un vector,
    una gráfica, una lista o un mensaje.

    Ejemplo: Suma de una progresión aritmética
    \begin{verbatim}
    > suma <- function(a1,d,n){
    +         an <- a1+(n-1)*d;
    +         ((a1+an)*n)/2}
    \end{verbatim}
\end{itemize}

\newslide

\subsection{Argumentos}

\begin{itemize}
    \item Una función con cuatro argumentos
    \begin{verbatim}
    > una.f <- function(a,b,c = 4,d = FALSE){x1<-a*z ...}
    \end{verbatim}

    \item Los argumentos \verb"a" y \verb"b" tienen que darse en el
    orden debido o, si los nombramos, podemos darlos en cualquier orden:
    \begin{verbatim}
    > una.f(4, 5)
    > una.f(b = 5, a = 4)
    \end{verbatim}

    \item Pero los argumentos con nombre siempre se tienen que dar
    después de los posicionales:
    \begin{verbatim}
    > una.f(c = 25, 4, 5) # error
    \end{verbatim}

    \item Los argumentos \verb"c" y \verb"d" tienen valores por defecto.
    Podemos especificarlos nosotros o no (i.e., usar los valores por defecto).

    \newslide

    \item \verb"args(nombre.funcion)" nos muestra los argumentos de cualquier función.

    \item ``\verb"..."'' permite pasar argumentos a otra función:
    \begin{verbatim}
    > f3 <- function(x, y, label = "la x", ...){
    +                plot(x, y, xlab = label, ...)}
    >
    > f3(1:5, 1:5)
    > f3(1:5, 1:5, col = "red")
    \end{verbatim}

    \item Para realizar funciones de dos variables se puede utilizar
    el comando \verb"outer". Por ejemplo:
    \begin{verbatim}
    > f <- function(x,y){cos(y)/(x^2-3)}
    > z <- outer(x,y,f)
    \end{verbatim}

\end{itemize}

\newslide

\subsection{Scope}

\begin{itemize}
    \item En la función \verb"una.f" ``\verb"z"'' es una ``free variable'':
    ¿cómo se especifica su valor? Lexical scoping. Ver documento
    \emph{Frames, environments and scope in R and S-PLUS} de J. Fox en
    \url{http://cran.r-project.org/doc/contrib/Fox-Companion/appendix.html}
    y sección 10.7 en \emph{An introduction to R}.
    También ver \verb"demo(scoping)".

    \item Un ejemplo
    \begin{verbatim}
    > cubo <- function(n) {
    + sq <- function() n*n # aquí n no es un argumento
    + n*sq()
    + }
    \end{verbatim}

    \item En esto R (lexical scope) y S-PLUS (static scope) son distintos.
\end{itemize}

\newslide

\subsection{Control de ejecución}
\begin{itemize}
    \item Principales instrucciones
    \begin{verbatim}
    if(cond) expr
    if(cond) cons.expr  else  alt.expr
    for(var in seq) expr
    while(cond) expr
    repeat expr
    break
    next
    \end{verbatim}

    \item La expresión \verb"expr" (también \verb"alt.expr")
    puede ser una expresión simple o una de las llamadas
    expresiones compuestas, normalmente del tipo \verb"{expr1; expr2}".

    \item Uno de los errores más habituales es el olvido de los corchetes
    \verb"{...}" alrededor de las instrucciones, i.e. después de \verb"if(...)"
    o \verb"for(...)".

    \item \verb"if (cond.logica)" \textit{instrucción} \verb"else"
    \textit{instrucción.alternativa}
    \begin{verbatim}
    > f4 <- function(x) {
    + if(x > 5) print("x > 5")
    + else {
    + y <- runif(1)
    + print(paste("y is ", y))
    + }
    + }
    \end{verbatim}

    \item \verb"ifelse" es una versión vectorizada
    (Thomas Unternährer, R-help, 2003-04-17)
    \begin{verbatim}
    > odd.even <- function(x) {
    + ifelse(x %% 2 == 1, "Odd", "Even")
    + }
    > mtf <- matrix(c(TRUE, FALSE, TRUE, TRUE),
    + nrow = 2)
    > ifelse(mtf, 0, 1)
    \end{verbatim}

    \item \verb"for ("\textit{variable.loop} \verb"in" \textit{valores}\verb")"
    \textit{instrucción}
    \begin{verbatim}
    > for(i in 1:10) cat("el valor de i es", i, "\n")

    > continue.loop <- TRUE
    > x <- 0
    > while(continue.loop) {
    + x <- x + 1
    + print(x)
    + if( x > 10) continue.loop <- FALSE
    + }
    \end{verbatim}

    \item \verb"break" para salir de un loop.

\end{itemize}

\newslide

\subsection{Cuando algo va mal}
\begin{itemize}
    \item Cuando se produce un error, \verb"traceback()"
    nos informa de la secuencia de llamadas antes del ``crash''
    de nuestra función.
    Es útil cuando se producen mensajes de error incomprensibles.

    \item Cuando se producen errores o la función da resultados
    incorrectos o ``warnings'' indebidos podemos seguir la ejecución de la función.

    \item \verb"browser" interrumpe la ejecución a partir de
    ese punto y permite seguir la ejecución o examinar el entorno;
    con ``n'' paso a paso, si otra tecla sigue la ejecución normal.
    ``Q'' para salir.

    \item \verb"debug" es como poner un \verb"broswer" al
    principio de la función y se ejecuta la función paso a paso.
    Se sale con ``Q''.
    \begin{verbatim}
    > debug(my.buggy.function)
    > ...
    > undebug(my.buggy.function)
    \end{verbatim}

\end{itemize}

\newslide

Ejemplo:
\begin{verbatim}
> my.f2 <- function(x, y) {
+ z <- rnorm(10) + y2 <- z * y + y3 <- z * y * x + return(y3 + 25)
+ }
> my.f2(runif(3), 1:4)
> debug(my.f2)
> my.f2(runif(3), 1:4)
> undebug(my.f2)
> # insertar un browser() y correr de nuevo
\end{verbatim}

\newslide

\subsection{Ejecución no interactiva}

\begin{itemize}
\item Con \verb"source" abrimos una sesión de R y hacemos
\begin{verbatim}
> source("mi.fichero.con.codigo.R")
\end{verbatim}

\item Con BATCH:
\begin{verbatim}
Rcmd BATCH mi.fichero.con.codigo.R
\end{verbatim}

\item \verb"source" es en ocasiones más útil porque informa
inmediatamente de errores en el código. BATCH no informa, pero no
requiere tener abierta una sesión (se puede correr en el
background).

\item Ver la ayuda: \verb"Rcmd BATCH --help"

\item Puede que necesitemos explícitos \verb"print" statements o
hacer \verb"source(my.file.R, echo = TRUE)".

\item \verb"sink" es el inverso de \verb"source" (lo manda todo a
un fichero).

\item Se pueden crear paquetes, con nuestras funciones, que se
comporten igual que los demás paquetes. Ver \emph{Writing R
extensions}.

\item R puede llamar código compilado en C/C++ y FORTRAN. Ver
\verb".C", \verb".Call", \verb".Fortran".

\item ``Lexical scoping'' importante en programación más avanzada.

\item No hemos mencionado el ``computing on the language'' (ej.,
\verb"do.call", \verb"eval", etc.).

\item R es un verdadero ``object-oriented language''. Dos
implementaciones, las S3 classes y las S4 classes.

\end{itemize}

\endinput
